\documentclass[12pt]{article}
\usepackage{geometry}                % See geometry.pdf to learn the layout options. There are lots.
\geometry{letterpaper}                   % ... or a4paper or a5paper or ... 
\usepackage{graphicx}
\usepackage{amssymb}
\usepackage{amsthm}
\usepackage{epstopdf}
\usepackage[utf8]{inputenc}
\usepackage[usenames,dvipsnames]{color}
\usepackage[table]{xcolor}
\usepackage{hyperref}
\DeclareGraphicsRule{.tif}{png}{.png}{`convert #1 `dirname #1`/`basename #1 .tif`.png}

\theoremstyle{definition}
\newtheorem{example}{Example}

\newenvironment{explanation}{%
   \setlength{\parindent}{0pt}
   \itshape
   \color{blue}
}{}

\newcommand{\projectname}{PanSim}
\newcommand{\productname}{Pandemic Simulator}
\newcommand{\projectleader}{O. Dominik}
\newcommand{\documentstatus}{In process}
%\newcommand{\documentstatus}{Submitted}
%\newcommand{\documentstatus}{Released}
\newcommand{\version}{V. 1.0}

\begin{document}
\begin{titlepage}
\begin{flushright}
\includegraphics[scale=.5]{htlleondinglogo.png}\\
\end{flushright}

\vspace{10em}

\begin{center}
{\Huge Project Proposal} \\[3em]
{\LARGE \productname} \\[3em]
\end{center}

\begin{flushleft}
\begin{tabular}{|l|l|}
\hline
Project Name & \projectname \\ \hline
Project Leader & \projectleader \\ \hline
Document state & \documentstatus \\ \hline
Version & \version \\ \hline
\end{tabular}
\end{flushleft}

\end{titlepage}
\section*{Revisions}
\begin{tabular}{|l|l|l|}
\hline
\cellcolor[gray]{0.5}\textcolor{white}{Date} & \cellcolor[gray]{0.5}\textcolor{white}{Author} & \cellcolor[gray]{0.5}\textcolor{white}{Change} \\ \hline
November 03, 2011&P. Bauer/T. Stütz&First version \\ \hline
\end{tabular}
\pagebreak

\tableofcontents
\pagebreak

\section{Introduction}

Our team decided to simulate a pandemic to help contain future panedemics more swiftly.
During the current corona crisis it has become apparent that we need some way to predict the direction of a pandemic.
In terms of feasebility we would first simulate a virus with just numbers and no graphical elements but later on we will incorperate a GUI if it is necessary.
Since this is a purely scientific simulation done by students affordability and market and economic efficiency can not be determined.
Since there isn't any funding involved in this project there are no risks only opportunities.


\pagebreak

\section{Initial Situation}

The current corona crisis has shown us that an pandemic cannot be foreseen and needs to be dealt with as fast as possible in order to stop its spread.
Currently measures like masks, shields or curfews cannot be tested unless they have shown some effect to stop the spread of a virus.
Currently, there are many diffrent simulations that try to predict the future of the corona crisis,
and only the corona crisis. The austrian government provides weekly predections on the corona virus based on the current situation in austria, from sources like that its also possible to update the simulation and guarantee a more precise outcome of the data.
Those simulations are made to try to find out when the pandemic will finaly end,
so they do not offer the possability to change the properties of the virus or to add new measures to test those.
Also many of the simulations that do offer such settings are really hard to read and do not explain their graphs in enough detail.

\pagebreak

\section{General Conditions and Constraints}

Our data needs to be obtained through trustworthy sources like government websites, universities etc.
This includes data on the virus as well as how to simulate viruses.
If we arent able to do that the whole simulation will give imprecise and false results.
So it is very necessary to get our informations from officaly and trustworthy sources. 
Also we currently lack the expertise to do such simulations which means we need to do some research first.
Otherwise we only create a simulation of how we think a simulation works and not how it should really work based on real data, events, etc.
The only tool we can currently use are our own laptops which means the simulation can't be too complex so that it doesn't run on these devices.

\pagebreak

\section{Project Objectives and System Concepts}

Our results shall be presented as clear and easy readable as possible, 
that means that data like infectionsrate, susceptible, infected, casualties and recovered should be displayed on a graph.
Calculations of the virus shall reflect a real and clear scenario with as much precision as possible.
Therefore all initial data for the virus, event probability, etc. must be displayed.

\pagebreak
\section{Opportunities and Risks}

The project has the following opportunities:
\begin{itemize}
\item Pandemics can be shown simplyfied.
\item Gain intel on an incoming pandemic.
\item Government support because it helps them.
\item Because of the high attention it shall be communicated very well at public events of the HTL Leonding
\end{itemize}

There are no risks because it is a purely scientific project without funding done by students.

\pagebreak
\section{Planning}

List of major project milestones
\begin{itemize}
\item All knowledge has been acquired. Around the start of january.
\item Running prototype with first results. Around the end of january.
\item Output in the form of a graph. One week after a running prototype.
\item Data acquisition from existing viruses. Around march.
\item Scientific results for predicting outbreaks. Around May.
\item Graphical user interface. Around june if there is some time left.
\end{itemize}

Our project will start after the required knowledge as been acquired which is estimated to be around january and will end with the with the start of summer break.
First prototype should be available around the end of january.
Implementation will start after knowledge acquisition.
The core engine of the simulation will be the biggest block of work to be done.
We are positive that we can do all the work in the given time.

\end{document}  